
%%%%%%%%%%%%%%%%%%%%%%%%%%%%%%%%%%%%%%%%%
% Programming/Coding Assignment
% LaTeX Template
%
% This template has been downloaded from:
% http://www.latextemplates.com
%
% Original author:
% Ted Pavlic (http://www.tedpavlic.com)
%
% Note:
% The \lipsum[#] commands throughout this template generate dummy text
% to fill the template out. These commands should all be removed when 
% writing assignment content.
%
% This template uses a Perl script as an example snippet of code, most other
% languages are also usable. Configure them in the "CODE INCLUSION 
% CONFIGURATION" section.
%
%%%%%%%%%%%%%%%%%%%%%%%%%%%%%%%%%%%%%%%%%

%----------------------------------------------------------------------------------------
%	PACKAGES AND OTHER DOCUMENT CONFIGURATIONS
%----------------------------------------------------------------------------------------



\documentclass{article}

\usepackage{fancyhdr} % Required for custom headers
\usepackage{lastpage} % Required to determine the last page for the footer
\usepackage{extramarks} % Required for headers and footers
\usepackage[usenames,dvipsnames]{color} % Required for custom colors
\usepackage{graphicx} % Required to insert images
\usepackage{listings} % Required for insertion of code
\usepackage{courier} % Required for the courier font
\usepackage{lipsum} % Used for inserting dummy 'Lorem ipsum' text into the template
\usepackage{setspace}
\usepackage{color}
\usepackage{comment}
\usepackage{caption}

\usepackage{hyperref}
\usepackage{natbib}
\usepackage{underscore}
\usepackage{subfigure}
\usepackage{fixltx2e}

\hypersetup{
    colorlinks=true,
    linkcolor=blue,
    filecolor=magenta,      
    urlcolor=cyan,
    breaklinks=true
}

\usepackage[]{algorithm2e}
\usepackage{pdfpages}
\usepackage{tikz}




%For python inclusion (http://widerin.org/blog/syntax-highlighting-for-python-scripts-in-latex-documents)
\definecolor{Code}{rgb}{0,0,0}
\definecolor{Decorators}{rgb}{0.5,0.5,0.5}
\definecolor{Numbers}{rgb}{0.5,0,0}
\definecolor{MatchingBrackets}{rgb}{0.25,0.5,0.5}
\definecolor{Keywords}{rgb}{0,0,1}
\definecolor{self}{rgb}{0,0,0}
\definecolor{Strings}{rgb}{0,0.63,0}
\definecolor{Comments}{rgb}{0,0.63,1}
\definecolor{Backquotes}{rgb}{0,0,0}
\definecolor{Classname}{rgb}{0,0,0}
\definecolor{FunctionName}{rgb}{0,0,0}
\definecolor{Operators}{rgb}{0,0,0}
\definecolor{Background}{rgb}{0.98,0.98,0.98}

% Margins
\topmargin=-0.45in
\evensidemargin=0in
\oddsidemargin=0in
\textwidth=6.5in
\textheight=9.0in
\headsep=0.25in

\linespread{1.1} % Line spacing

% Set up the header and footer
\pagestyle{fancy}
\lhead{\hmwkAuthorName} % Top left header
\chead{\hmwkClass\ (\hmwkClassInstructor\ \hmwkClassTime): \hmwkTitle} % Top center head
\chead{\hmwkClass\ (\hmwkClassInstructor): \hmwkTitle} % Top center head
\rhead{\firstxmark} % Top right header
\lfoot{\lastxmark} % Bottom left footer
\cfoot{} % Bottom center footer
\rfoot{Page\ \thepage\ of\ \protect\pageref{LastPage}} % Bottom right footer
\renewcommand\headrulewidth{0.4pt} % Size of the header rule
\renewcommand\footrulewidth{0.4pt} % Size of the footer rule

\setlength\parindent{0pt} % Removes all indentation from paragraphs

%----------------------------------------------------------------------------------------
%	CODE INCLUSION CONFIGURATION
%----------------------------------------------------------------------------------------

\definecolor{MyDarkGreen}{rgb}{0.0,0.4,0.0} % This is the color used for comments
\lstloadlanguages{Perl} % Load Perl syntax for listings, for a list of other languages supported see: ftp://ftp.tex.ac.uk/tex-archive/macros/latex/contrib/listings/listings.pdf
\lstset{language=Perl, % Use Perl in this example
        frame=single, % Single frame around code
        basicstyle=\small\ttfamily, % Use small true type font
        keywordstyle=[1]\color{Blue}\bf, % Perl functions bold and blue
        keywordstyle=[2]\color{Purple}, % Perl function arguments purple
        keywordstyle=[3]\color{Blue}\underbar, % Custom functions underlined and blue
        identifierstyle=, % Nothing special about identifiers                                         
        commentstyle=\usefont{T1}{pcr}{m}{sl}\color{MyDarkGreen}\small, % Comments small dark green courier font
        stringstyle=\color{Purple}, % Strings are purple
        showstringspaces=false, % Don't put marks in string spaces
        tabsize=5, % 5 spaces per tab
        %
        % Put standard Perl functions not included in the default language here
        morekeywords={rand},
        %
        % Put Perl function parameters here
        morekeywords=[2]{on, off, interp},
        %
        % Put user defined functions here
        morekeywords=[3]{test},
       	%
        morecomment=[l][\color{Blue}]{...}, % Line continuation (...) like blue comment
        numbers=left, % Line numbers on left
        firstnumber=1, % Line numbers start with line 1
        numberstyle=\tiny\color{Blue}, % Line numbers are blue and small
        stepnumber=5 % Line numbers go in steps of 5
}

% Creates a new command to include a perl script, the first parameter is the filename of the script (without .pl), the second parameter is the caption
\newcommand{\perlscript}[2]{
\begin{itemize}
\item[]\lstinputlisting[caption=#2,label=#1]{#1.pl}
\end{itemize}
}


%----------------------------------------------------------------------------------------
%	DOCUMENT STRUCTURE COMMANDS
%	Skip this unless you know what you're doing
%----------------------------------------------------------------------------------------

% Header and footer for when a page split occurs within a problem environment
\newcommand{\enterProblemHeader}[1]{
\nobreak\extramarks{#1}{#1 continued on next page\ldots}\nobreak
\nobreak\extramarks{#1 (continued)}{#1 continued on next page\ldots}\nobreak
}

% Header and footer for when a page split occurs between problem environments
\newcommand{\exitProblemHeader}[1]{
\nobreak\extramarks{#1 (continued)}{#1 continued on next page\ldots}\nobreak
\nobreak\extramarks{#1}{}\nobreak
}

\setcounter{secnumdepth}{0} % Removes default section numbers
\newcounter{homeworkProblemCounter} % Creates a counter to keep track of the number of problems

\newcommand{\homeworkProblemName}{}
\newenvironment{homeworkProblem}[1][Problem \arabic{homeworkProblemCounter}]{ % Makes a new environment called homeworkProblem which takes 1 argument (custom name) but the default is "Problem #"
\stepcounter{homeworkProblemCounter} % Increase counter for number of problems
\renewcommand{\homeworkProblemName}{#1} % Assign \homeworkProblemName the name of the problem
\section{\homeworkProblemName} % Make a section in the document with the custom problem count
\enterProblemHeader{\homeworkProblemName} % Header and footer within the environment
}{
\exitProblemHeader{\homeworkProblemName} % Header and footer after the environment
}

\newcommand{\problemAnswer}[1]{ % Defines the problem answer command with the content as the only argument
\noindent\framebox[\columnwidth][c]{\begin{minipage}{0.98\columnwidth}#1\end{minipage}} % Makes the box around the problem answer and puts the content inside
}

\newcommand{\homeworkSectionName}{}
\newenvironment{homeworkSection}[1]{ % New environment for sections within homework problems, takes 1 argument - the name of the section
\renewcommand{\homeworkSectionName}{#1} % Assign \homeworkSectionName to the name of the section from the environment argument
\subsection{\homeworkSectionName} % Make a subsection with the custom name of the subsection
\enterProblemHeader{\homeworkProblemName\ [\homeworkSectionName]} % Header and footer within the environment
}{
\enterProblemHeader{\homeworkProblemName} % Header and footer after the environment
}

%----------------------------------------------------------------------------------------
%	NAME AND CLASS SECTION
%----------------------------------------------------------------------------------------

\newcommand{\hmwkTitle}{Assignment\ \#1 } % Assignment title
\newcommand{\hmwkDueDate}{Thursday,\ January\ 31,\ 2019} % Due date
\newcommand{\hmwkClass}{Web Science} % Course/class
%\newcommand{\hmwkClassTime}{10:30am} % Class/lecture time
\newcommand{\hmwkClassInstructor}{Alexander Nwala} % Teacher/lecturer
\newcommand{\hmwkAuthorName}{Apurva Modi} % Your name

%----------------------------------------------------------------------------------------
%	TITLE PAGE
%----------------------------------------------------------------------------------------

\title{
\vspace{2in}
\textmd{\textbf{\hmwkClass:\ \hmwkTitle}}\\
%\normalsize\vspace{0.1in}\small{Due\ on\ \hmwkDueDate}\\
%\vspace{0.1in}\large{\textit{\hmwkClassInstructor\ \hmwkClassTime}}
\vspace{0.1in}\large{\textit{\hmwkClassInstructor}}
\vspace{3in}
}

\author{\textbf{\hmwkAuthorName}}
\date{Sunday, January 28, 2018} % Insert date here if you want it to appear below your name

%----------------------------------------------------------------------------------------

\begin{document}

\maketitle
\newpage



%----------------------------------------------------------------------------------------
%	TABLE OF CONTENTS
%----------------------------------------------------------------------------------------

%\setcounter{tocdepth}{1} % Uncomment this line if you don't want subsections listed in the ToC

\newpage
\tableofcontents
\newpage

%----------------------------------------------------------------------------------------
%	PROBLEM 1
%----------------------------------------------------------------------------------------

% To have just one problem per page, simply put a \clearpage after each problem

\begin{homeworkProblem}

Demonstrate that you know how to use "curl" well enough to correctly POST data to a form.Show that the HTML response that is returned is "correct".That is, the server should take the arguments you POSTed and\\
build a response accordingly.  Save the HTML response to a file and then view that file in a browser and take a screen shot.\\

%\problemAnswer{
 \textbf{SOLUTION :}\\
  \begin{verbatim}
  curl -i -d "fname=APURVA&lname=MODI" -X POST 
  http://www.cs.odu.edu/~anwala/files/temp/namesEcho.php
  \end{verbatim}
 
\begin{figure}[h]
  \centering
    \includegraphics[width=1.1\textwidth,height=0.6\textheight]{figure-curl-post}
      \caption{Data posting with curl}
\end{figure}
 
%}

\end{homeworkProblem}
\clearpage
\newpage
%----------------------------------------------------------------------------------------
%   PROBLEM 2
%----------------------------------------------------------------------------------------

\begin{homeworkProblem}
 
Write a Python program that:\\
\begin{enumerate}
\item \textbf{} takes as a command line argument a web page extracts all the links from the page
\item \textbf{} lists all the links that result in PDF files, and prints out the bytes for each of the links.  \\
			(note: be sure to follow all the redirects until the link terminates with a "200 OK".)\\
			show that the program works on 3 different URIs, one of which needs to be:\\
		     \url{http://www.cs.odu.edu/~mln/teaching/cs532-s17/test/pdfs.html}\\
 \end{enumerate}
%\problemAnswer{
 \textbf{SOLUTION}\\\\
 The solution for this problem is outlined by the following steps:\\
 
 \begin{enumerate}
   \item \textbf{Extracting the Links from the Web Pages :} 

 \begin{figure}[h]
  \centering
    \includegraphics[width=1.1\textwidth,height=0.6\textheight]{a11}
\end{figure}

\begin{figure}[h]
  \centering
    \includegraphics[width=1.1\textwidth,height=0.6\textheight]{a21}
      \caption{Command Line Argument}
\end{figure}

 \begin{lstlisting}[language=Python, caption=A1.py]
from bs4 import BeautifulSoup
import requests
url = input("Enter the search query :");

re=requests.get("http://" + url);
data =re.text
soup =BeautifulSoup(data)
file = open("pdfurl.txt","w") 
for link in soup.find_all('a'):
    line = link.get('href')
    if "http" not in line :
        continue
    else:
        response = requests.get(line)
        if (response.headers['content-type'] == 'application/pdf'):
            hrefLinks=link.get('href')
            contentLength =response.headers['content-length']
            file.write(hrefLinks+"\n"+contentLength + " bytes\n")
            
file.close()
\end{lstlisting}

\item \textbf{ The above code when runs with test URI and Adds PDF links to pdfurl.txt file with size in bytes.}

\begin{lstlisting}[caption=PDF files with their size]
http://www.cs.odu.edu/~mln/pubs/ht-2015/hypertext-2015-temporal-violations.pdf
2184076 bytes
http://www.cs.odu.edu/~mln/pubs/tpdl-2015/tpdl-2015-annotations.pdf
622981 bytes
http://arxiv.org/pdf/1512.06195
1748959 bytes
http://www.cs.odu.edu/~mln/pubs/tpdl-2015/tpdl-2015-off-topic.pdf
4308768 bytes
http://www.cs.odu.edu/~mln/pubs/tpdl-2015/tpdl-2015-stories.pdf
1274604 bytes
http://www.cs.odu.edu/~mln/pubs/tpdl-2015/tpdl-2015-profiling.pdf
639001 bytes
http://www.cs.odu.edu/~mln/pubs/jcdl-2014/jcdl-2014-brunelle-damage.pdf
2205546 bytes
http://bit.ly/1ZDatNK
720476 bytes
http://www.cs.odu.edu/~mln/pubs/jcdl-2015/jcdl-2015-mink.pdf
1254605 bytes
http://www.cs.odu.edu/~mln/pubs/jcdl-2015/jcdl-2015-arabic-sites.pdf
709420 bytes
http://www.cs.odu.edu/~mln/pubs/jcdl-2015/jcdl-2015-dictionary.pdf
2350603 bytes

\end{lstlisting}

\end{enumerate}
\end{homeworkProblem}
\clearpage
\newpage

%----------------------------------------------------------------------------------------
%   PROBLEM 3
%----------------------------------------------------------------------------------------

\begin{homeworkProblem}
Consider the "bow-tie" graph in the Broder et al. paper (fig 9): "\textbf{http://www9.org/w9cdrom/160/160.html}\\  
Now consider the following graph:
\begin{verbatim}
    A --> B
    B --> C
    C --> D
    C --> A
    C --> G
    E --> F
    G --> C
    G --> H
    I --> H
    I --> K
    L --> D
    M --> A
    M --> N
    N --> D
    O --> A
    P --> G
\end{verbatim}
For the above graph, give the values for:
\begin{verbatim}
    IN: 
    SCC: 
    OUT: 
    Tendrils: 
    Tubes: 
    Disconnected:
\end{verbatim}

\clearpage
%\problemAnswer{

 \textbf{SOLUTION}
 
 \begin{figure}[h]
  \centering
    \includegraphics[width=1.1\textwidth,height=0.6\textheight]{a3}
      \caption{Demonstration of the graph}
\end{figure}



\begin{verbatim}
IN: M , O , P
SCC: C , B , G , A
OUT:  D , H 
Tendrils: K , I , L
Tubes: N
Disconnected: E , F

\end{verbatim}
     
   
%}

\end{homeworkProblem}
\clearpage
\newpage

\bibliographystyle{plain}
\bibliography{A1bibFile}

\end{document}
